
\newif\iflineno
\documentclass[5p,authoryear,preprint]{elsarticle}\linenofalse
%\documentclass[preprint,review,authoryear,12pt]{elsarticle}\linenotrue

\usepackage{amsmath,amssymb,bm,enumitem,cases,calc,pgfplots}
\usepackage{lineno}

\journal{ISA Transactions}

\begin{document}

\begin{frontmatter}

\title{Three-dimensional bearing-only circumnavigation 
	using fully actuated system approaches}

\author[hit]{Shida Cao}
\ead{shida_cao@163.com}
\author[sust,hit]{Guangren Duan\corref{cor1}}
\ead{g.r.duan@hit.edu.cn}

\cortext[cor1]{Corresponding author}

\affiliation[hit]{
	organization={Center for Control Theory and Guidance Technology},
	addressline={Harbin Institute of Technology}, 
	city={Harbin},
	postcode={150001}, 
	state={Heilongjiang},
	country={China}}

\affiliation[sust]{
	organization={Shenzhen Key Laboratory of Control Theory and Intelligent Systems},
	addressline={Southern University of Science and Technology},
	city={Shenzhen},
	postcode={518055},
	state={Guangdong},
	country={China}}

\begin{abstract}
Abstract text.
\end{abstract}

\begin{keyword}
Fully actuated system\sep 
Multi-agent systems\sep 
Three-dimensional circumnavigation\sep 
Bearing-only measurement\sep
PX4\sep
Co-simulation
\end{keyword}

\end{frontmatter}

\iflineno\linenumbers\else\fi

\allowdisplaybreaks[4]
\graphicspath{{fig}}

%%%%%%%%%% theorem %%%%%%%%%%
\newdefinition{assumption}{Assumption}
\newdefinition{problem}{Problem}
\newdefinition{remark}{Remark}
\newtheorem{proposition}{Proposition}
\newtheorem{theorem}{Theorem}
\newtheorem{lemma}[theorem]{Lemma}
\newproof{proof}{Proof}

%%%%%%%%%% user defined signs %%%%%%%%%%
\def\alI{a_l^1}
\def\alO{a_l^0}
\def\arhoI{a_\rho^1}
\def\arhoO{a_\rho^0}
\def\aomega{a_{\omega}}
\def\avII{a_{v_2}}
\def\B{\bm B}
\def\intd{\,\mathrm{d}}
\def\ddt{\frac{\mathrm d}{\mathrm dt}}
\def\erho{e_\rho}
\def\erhodot{\dot{e}_\rho}
\def\erhoddot{\ddot{e}_\rho}
\def\erhohat{\hat{e}_\rho}
\def\evII{e_{v_2}}
\def\evIIdot{\dot{e}_{v_2}}
\def\fai{\bm{\varphi}}
\def\faidot{\dot{\bm{\varphi}}}
\def\faiT{\bm{\varphi}^\mathrm{T}}
\def\faibar{\bar{\bm{\varphi}}}
\def\faibarT{\bar{\bm{\varphi}}^\mathrm{T}}
\def\faibardot{\dot{\bar{\fai}}}
\def\ldian{\dot{l}}
\def\lddot{\ddot{l}}
\def\n{\bm{n}}
\def\nT{\bm{n}^\mathrm{T}}
\def\omegaII{\omega_2}
\def\omegaIII{\omega_3}
\def\omegaIIIdot{\dot{\omega}_3}
\def\rhodot{\dot{\rho}}
\def\rhoddot{\ddot{\rho}}
\def\rhohat{\hat{\rho}}
\def\rhohatdot{\dot{\hat{\rho}}}
\def\rhotilde{\tilde{\rho}}
\def\rhotildedot{\dot{\tilde{\rho}}}
\def\R{\mathbb{R}}
\def\RIII{\mathbb{R}^3}
\def\SII{\mathbb{S}^2}
\def\T{\mathrm{T}}
% tau1
\def\tauI{\bm{\tau}_1}
\def\tauIdot{\dot{\bm{\tau}}_1}
\def\tauIdotT{\dot{\bm{\tau}}_1^\mathrm{T}}
\def\tauIT{\bm{\tau}_1^\mathrm{T}}
% tau2
\def\tauII{\bm{\tau}_2}
\def\tauIIdot{\dot{\bm{\tau}}_2}
\def\tauIIdotT{\dot{\bm{\tau}}_2^\mathrm{T}}
\def\tauIIT{\bm{\tau}_2^\mathrm{T}}
% tau3
\def\tauIII{\bm{\tau}_3}
\def\tauIIIdot{\dot{\bm{\tau}}_3}
\def\tauIIIdotT{\dot{\bm{\tau}}_3^\mathrm{T}}
\def\tauIIIT{\bm{\tau}_3^\mathrm{T}}
% tau23
\def\tauIIiii{\bm{\tau}_{23}}
\def\tauIIiiiT{\bm{\tau}_{23}^\mathrm{T}}
\def\thetadot{\dot{\theta}}
\def\thetaddot{\ddot{\theta}}
\def\vbar{\bar{v}}
\def\Vdot{\dot{V}}
\def\vI{v_1}
\def\vIdot{\dot{v}_1}
\def\vII{v_2}
\def\vIIdot{\dot{v}_2}
\def\vIII{v_3}
\def\vIIIdot{\dot{v}_3}
\def\x{\bm{x}}
\def\xdot{\dot{\bm{x}}}
\def\xhat{\hat{\bm{x}}}
\def\xhatdot{\dot{\hat{\bm{x}}}}
\def\xtilde{\tilde{\bm{x}}}
\def\xtildedot{\dot{\tilde{\bm{x}}}}
\def\xtildeT{\tilde{\bm{x}}^\mathrm{T}}
\def\y{\bm{y}}
\def\ydot{\dot{\bm{y}}}
\def\yddot{\ddot{\bm{y}}}
\def\vbar{\bar{v}}
\def\vbardot{\dot{\bar{v}}}

\section{Introduction}

Contributions:
\begin{enumerate}
\item A three-dimensional scalar estimator is designed.

\item A mixed-order FAS controller is designed.

\item A numerical-PX4 co-simulation is conducted to verify our control strategy.

\end{enumerate}






\emph{Notations}:
For scalars, let $\R$, $\R_+$ and $\R_{\ge0}$ denote the sets of real numbers, positive numbers and nonnegative real numbers.
For vectors, we use $\RIII$ and $\SII$ to denote the sets of 3-dimensional column vectors and 3-dimensional unit column vectors.
For matrices, the notation $\mathrm{SO}(3)$ is used to represent the set of $3\times3$ rotation matrices.
$\bm0$ denotes the ordinary point of $\R^3$.
$\bm I$ denotes the $3\times3$ rotation matrix.
${\|\cdot\|}$ denotes the vector 2-norm.
Unless defined in the first time, we usually omit the dependence of variables 
on the time $t$, for example, $a$ represents $a(t)$.
To distinguish between scalars and vectors, 3-dimensional vectors and $3\times3$ matrices are in bold font.

\section{Problem statement and FAS model construction}

In this section, we introduce our 3-dimensional BOC problem and finally derive its FAS model.
For the model derivation, the dynamics of the space vector basis is important.

\subsection{Problem statement}

In this subsection, we describe the 3-dimensional BOC problem and then give it a formal statement.

In our multi-agent system, there is one stationary target $\x\in\RIII$ and one agent $\y(t)\in\RIII$ whose acceleration $\yddot$ is controlled.
The agent moves in 3-dimensional space and is supposed to perform a circumnavigation task around the target, i.e., to move around the target in a planar circular orbit with desired distance $d\in\R_+$ and angular speed $\alpha\in\R_+$.
Since the circular orbit is in a plane and the agent moves in the space, we denote the orbit's unit normal vector by $\n\in\SII$ and suppose that the agent can know this normal vector.
The desired orbit is illustrated in \figurename~\ref{fig:desired orbit}.

\begin{figure}[!htbp]
\centering
I\includegraphics[width=0.6\linewidth]{fig/example.png}
\caption{The desired orbit.}
\label{fig:desired orbit}
\end{figure}

Unlike general circumnavigation problems, the uniqueness of the BOC problem lies in the fact that the agent-target distance $\rho(t)=\|\x-\y(t)\|\in\R_{\ge0}$ can not be measured by the sensors equipped in the agent.
As for the target's information, what can be measured is only the bearing information, 
that is, the direction from the agent to the target.
Mathematically, this direction is represented by the radial unit vector $\tauI(t)=(\x-\y)/\|\x-\y\|\in\SII$ which points from the agent to the target.

With the normal vector $\n$, we can define a vector basis together with the bearing unit vector $\tauI$.
Denote the tangential unit vectors $\tauII(t)=[\n\times\tauI(t)]/\|\n\times\tauI(t)\|\in\SII$,
$\tauIII(t)=\tauI(t)\times\tauII(t)\in\SII$.
With this basis, we can define the agent's tangential speeds $\vII(t)=\tauIIT(t)\ydot(t)\in\R$, $\vIII(t)=\tauIIIT(t)\ydot(t)\in\R$.
Also, the agent's orbital angular speed $\omegaII(t)=\vII(t)/\rho(t)\in\R$ and the non-orbital angular speed $\omegaIII(t)=\vIII(t)/\rho(t)\in\R$ can be defined.

For formal problem description, the following definitions should be introduced:
Use $\erho(t)=\rho(t)-d\in\R$ to denote the tracking error of the agent-target distance;
$\evII(t)=\vII(t)-\alpha\in\R$ to denote the tracking error of the agent's orbital tangential speed;
and $\theta(t)=\arccos(\nT\tauI(t))-\pi/2\in(-\pi,\pi)$ to denote the latitude angle of the agent to the target.

After all of the above introduction, finally we give a formal statement of our 3-dimensional BOC problem.

\begin{problem}\label{prb:BOC}
With the knowledge of the target's position~$\y$, the target's velocity~$\ydot$, the agent's angular speeds~$\omega_2$, $\omega_3$, the unit vectors~$\tauI$, $\tauII$, $\tauIII$, and the orbit's unit normal vector~$\n$, design the agent's acceleration~$\yddot$ to achieve that
\[
\lim_{t\to\infty}\erho(t)=\lim_{t\to\infty}\evII(t)=\lim_{t\to\infty}\theta(t)=0.
\]
\end{problem}

\subsection{Space vector basis}

In this subsection, we explore the dynamics of the space vector basis $\tauI$, $\tauII$ and $\tauIII$.

To obtain derivative of $\tauI$, these two propositions are presented.

\begin{proposition}\label{prp:h/|h| dot=?}
For a vector-valued function $\bm h(t)\colon\R_{\ge0}\to\R^3$, $\bm h(t)\ne\bm0$, the derivative of $\bm h(t)/\|\bm h(t)\|$ is
\[
\ddt\frac{\bm h(t)}{\|\bm h(t)\|}=\left(\bm I-\frac{\bm h(t)}{\|\bm h(t)\|}\frac{\bm h^\T(t)}{\|\bm h(t)\|}\right)\frac{\dot{\bm h}(t)}{\|\bm h(t)\|}.
\]
\end{proposition}

\begin{proposition}\label{prp:I=tau1*tau1T+tau2*tau2T+tau3*tau3T}
Let $\tauI$, $\tauII$ and $\tauIII$ defined as in the previous subsection, then
$I=\tauI\tauIT+\tauII\tauIIT+\tauIII\tauIIIT$.
\end{proposition}

Denote the agent's radial speed as $\vI(t)=\tauIT\ydot\in\R$.
We know there is $\ydot=\vI\tauI+\vII\tauII+\vIII\tauIII$.
Using Proposition~\ref{prp:h/|h| dot} and~\ref{prp:I=tau1*tau1T+tau2*tau2T+tau3*tau3T}, we know the derivative of the radial unit vector $\tauI$ is
\begin{equation}\label{eqn:tau1 dot=?}
\begin{aligned}
	\tauIdot&=\ddt\frac{\x-\y}{\|\x-\y\|}=(I-\tauI\tauIT)\frac{-\ydot}\rho\\
	&=(\tauII\tauIIT+\tauIII\tauIIIT)\frac{-(\vI\tauI+\vII\tauII+\vIII\tauIII)}\rho\\
	&=-\frac\vII\rho\tauII-\frac\vIII\rho\tauIII=-\omegaII\tauII-\omegaIII\tauIII.
\end{aligned}\end{equation}

Then we derive the derivative of the agent's tangential speed $\tauII$.
From the definition of the latitude angle $\theta$, we know
\[
\|\n\times\tauI\|=\sin(\arccos(\nT\tauI))=\sin\left(\theta+\frac\pi2\right)=\cos\theta.
\]
Noting that
\begin{gather*}
	\nT\tauI=\cos(\arccos(\nT\tauI))=\cos\left(\theta+\frac\pi2\right)=-\sin\theta,\\
	\nT\tauII=\nT\frac{\n\times\tauI}{\|\n\times\tauI\|}=0,
\end{gather*}
we get
\begin{align*}
	\n\times\tauIII&=\n\times(\tauI\times\tauII)=(\nT\tauII)\tauI-(\nT\tauI)\tauII\\
	&=-(\nT\tauI)\tauII=\sin\theta\tauII,
\end{align*}
Hence, with Proposition~\ref{prp:h/|h| dot=?} and~\ref{prp:I=tau1*tau1T+tau2*tau2T+tau3*tau3T}, we know the derivative of $\tauII$ is
\begin{equation}\label{eqn:tau2 dot=?}\begin{aligned}
	\tauIIdot&=\ddt\frac{\n\times\tauI}{\|\n\times\tauI\|}=(I-\tauII\tauIIT)\frac{\n\times\tauIdot}{\cos\theta}\\
	&=(\tauI\tauIT+\tauIII\tauIIIT)\frac{-\omegaII(\n\times\tauII)-\omegaIII(\n\times\tauIII)}{\cos\theta}\\
	&=(\tauI\tauIT+\tauIII\tauIIIT)\frac{-\omegaII(\n\times\tauII)}{\cos\theta}\\
	&=(\tauI\tauIT+\tauIII\tauIIIT)\frac{-\omegaII\left(\n\times\frac{\n\times\tauI}{\|\n\times\tauI\|}\right)}{\cos\theta}\\
	&=(\tauI\tauIT+\tauIII\tauIIIT)\frac{-\omegaII[\n\times(\n\times\tauI)]}{\cos^2\theta}\\
	&=(\tauI\tauIT+\tauIII\tauIIIT)\frac{-\omegaII(-\sin\theta\n-\tauI)}{\cos^2\theta}\\
	&=\frac\omegaII{\cos^2\theta}(\tauI\tauIT+\tauIII\tauIIIT)(\sin\theta\n+\tauI)\\
	&=\frac\omegaII{\cos^2\theta}(\cos^2\theta\tauI+\sin\theta\cos\theta\tauIII)\\
	&=\omegaII\tauI+\omegaII\tan\theta\tauIII.
\end{aligned}\end{equation}

For the derivation of the derivative of the agent's tangential speed $\tauIII$, the following proposition is introduced.

\begin{proposition}\label{prp:R dot=RS}
	For a matrix-valued function $\bm R(t)\colon\R_{\ge0}\to\mathrm{SO}(3)$, there is
	$\dot{\bm R}(t)=\bm R(t)\bm S(t)$,
	where $\bm S(t)\in\R^{3\times3}$ is an anti-symmetric matrix.
\end{proposition}

Recalling \eqref{eqn:tau1 dot=?} and \eqref{eqn:tau2 dot=?}, we know
\begin{align*}
	\begin{bmatrix}\tauIdot&\tauIIdot&\tauIIIdot\end{bmatrix}=
	\begin{bmatrix}\tauI&\tauII&\tauIII\end{bmatrix}
	\begin{bmatrix}
		0&\omegaII&*\\
		-\omegaII&0&*\\
		-\omegaIII&\omegaII\tan\theta&*
	\end{bmatrix},
\end{align*}
where each ``$*$'' represents an unknown term.
With Proposition~\ref{prp:R dot=RS}, we can obtain that
\begin{align*}
	&\begin{bmatrix}\tauIdot&\tauIIdot&\tauIIIdot\end{bmatrix}=\\
	&\qquad\begin{bmatrix}\tauI&\tauII&\tauIII\end{bmatrix}
	\begin{bmatrix}
		0&\omegaII&\omegaIII\\
		-\omegaII&0&-\omegaII\tan\theta\\
		-\omegaIII&\omegaII\tan\theta&0
	\end{bmatrix}.
\end{align*}
Therefore, the derivatives of the space vector basis $\tauI$, $\tauII$ and $\tauIII$ are as follows:
\begin{subnumcases}{}
	\label{eqn:tau1dot=?}
	\tauIdot=-\omegaII\tauII-\omegaIII\tauIII\\
	\label{eqn:tau2dot=?}
	\tauIIdot=\omegaII\tauI+\omegaII\tan\theta\tauIII\\
	\label{eqn:tau3dot=?}
	\tauIIIdot=\omegaIII\tauI-\omegaII\tan\theta\tauII.
\end{subnumcases}

\subsection{FAS model construction}

In this subsection, we deduce the FAS model of the 3-dimentional BOC problem.
Define the arc length $l=\rho\theta\in\R$.
Choose the three variables: the agent-target distance $\rho$, the agent's tangential speed $\vII$ and the arc length $l$ as system states.
In this section, we deduce the dynamics of these three system states.

Firstly, we derive the dynamics of the agent-target distance $\rho$ and the agent's tangential speed $\vII$.
To obtain the derivative of $\rho$.
The following proposition is presented.

\begin{proposition}\label{prp:|h| dot=?}
For a function $\bm h(t)\colon\R_{\ge0}\to\R^3$, the derivative of $\|\bm h(t)\|$ is
\[
\ddt\|\bm h(t)\|=\frac{\bm h^\T(t)}{\|\bm h(t)\|}\dot{\bm h}(t).
\]
\end{proposition}

With Proposition~\ref{prp:|h| dot=?}, taking the derivative of $\rho$, gives
\begin{equation}\label{eqn:rhodot=-v1}
	\rhodot=\ddt\|\x-\y\|=\tauIT(-\ydot)=-\vI.
\end{equation}
From \eqref{eqn:tau1dot=?}, taking the derivative again, we have
\begin{equation}\label{eqn:rhoddot=?}
	\begin{aligned}
		\rhoddot&=-\vIdot=-\ddt(\tauIT\ydot)=-\tauIdotT\ydot-\tauIT\yddot\\
		&=-(-\omegaII\tauII-\omegaIII\tauIII)^\T\ydot-\tauIT\yddot\\
		&=\omegaII\tauIIT\ydot+\omegaIII\tauIIIT\ydot-\tauIT\yddot\\
		&=\vII\omegaII+\vIII\omegaIII-\tauIT\yddot.
	\end{aligned}
\end{equation}
With \eqref{eqn:tau2dot=?}, taking the derivative of $\vII$, we know
\begin{equation}\begin{aligned}\label{eqn:v2dot=?}
		\vIIdot&=\ddt(\tauIIT\ydot)=\tauIIdotT\ydot+\tauIIT\yddot\\
		&=(\omegaII\tauI+\omegaII\tan\theta\tauIII)^\T\ydot+\tauIIT\yddot\\
		&=\omegaII\tauIT\ydot+\omegaII\tan\theta\tauIIIT\ydot+\tauIIT\yddot\\
		&=\vI\omegaII+\vIII\omegaII\tan\theta+\tauIIT\yddot.
\end{aligned}\end{equation}

Now it is time to take the derivative of the arc length $l$.
This needs some preparations.
Taking the derivative of $\theta$, yields
\begin{align}\label{eqn:thetadot=?}
	\thetadot=\ddt\left(\arccos(\nT\tauI)-\frac\pi2\right)=-\frac{\nT\tauIdot}{\cos\theta}=\omegaIII.
\end{align}
From \eqref{eqn:thetadot=?}, taking the derivative of $l$, gives
\begin{align*}
	\ldian=\ddt(\rho\theta)=\rhodot\theta+\rho\thetadot=\rhodot\theta+\rho\omegaIII=\rhodot\theta+\vIII.
\end{align*}
With \eqref{eqn:tau3dot=?}, taking the derivative of $\vIII$, gives
\begin{equation}\begin{aligned}\label{eqn:v3dot=?}
		\vIIIdot&=\ddt(\tauIIIT\ydot)=\tauIIIdotT\ydot+\tauIIIT\yddot\\
		&=(\omegaIII\tauI-\omegaII\tan\theta\tauII)^\T\ydot+\tauIIIT\yddot\\
		&=\omegaIII\tauIT\ydot-\omegaII\tan\theta\tauIIT\ydot+\tauIIIT\yddot\\
		&=\vI\omegaIII-\vII\omegaII\tan\theta+\tauIIIT\yddot.
\end{aligned}\end{equation}
From \eqref{eqn:rhoddot=?}, \eqref{eqn:thetadot=?}, \eqref{eqn:v3dot=?},
taking the second-order derivative of $l$, we can obtain
\begin{equation}\label{eqn:lddot=?}
	\begin{aligned}
		\lddot&=\rhoddot\theta+\rhodot\thetadot+\vIIIdot\\
		&=(\vII\omegaII+\vIII\omegaIII-\tauIT\yddot)\theta+(-\vI)\omegaIII\\
		&\qquad+(\vI\omegaIII-\vII\omegaII\tan\theta+\tauIIIT\yddot)\\
		&=\vII\omegaII\theta+\vIII\omegaIII\theta-\vII\omegaII\tan\theta+(\tauIIIT-\theta\tauIT)\yddot
\end{aligned}\end{equation}

With \eqref{eqn:rhoddot=?}, \eqref{eqn:v2dot=?}, \eqref{eqn:lddot=?},
we obtain the system equation as follows,
\[\begin{cases}
	\rhoddot=\vII\omegaII+\vIII\omegaIII-\tauIT\yddot\\
	\vIIdot=\vI\omegaII+\vIII\omegaII\tan\theta+\tauIIT\yddot\\
	\lddot=\vII\omegaII\theta+\vIII\omegaIII\theta-\vII\omegaII\tan\theta+(\tauIIIT-\theta\tauIT)\yddot.
\end{cases}\]
Hence, the error system equation is
\[
\begin{bmatrix}\erhoddot\\ \evIIdot\\ \lddot\end{bmatrix}
=\begin{bmatrix}
	\vII\omegaII+\vIII\omegaIII\\
	\vI\omegaII+\vIII\omegaII\tan\theta\\
	\vII\omegaII\theta+\vIII\omegaIII\theta-\vII\omegaII\tan\theta
\end{bmatrix}+
\begin{bmatrix}-\tauIT\\ \tauIIT\\ \tauIIIT-\theta\tauIT\end{bmatrix}\yddot.
\]
Let
\begin{gather*}
	\bm f(t)=\begin{bmatrix}
		\vII\omegaII+\vIII\omegaIII\\
		\vI\omegaII+\vIII\omegaII\tan\theta\\
		\vII\omegaII\theta+\vIII\omegaIII\theta-\vII\omegaII\tan\theta
	\end{bmatrix},\\
	\B(t)=\begin{bmatrix}-\tauIT\\ \tauIIT\\ \tauIIIT-\theta\tauIT\end{bmatrix}.
\end{gather*}
The error system equation can be rewritten as
\begin{equation}\label{eqn:error system}
	\begin{bmatrix}\erhoddot\\ \evIIdot\\ \lddot\end{bmatrix}=\bm f+\B\yddot.
\end{equation}
Considering that
\[
\B=\begin{bmatrix}-1&0&0\\ 0&1&0\\ -\theta&0&1\end{bmatrix}
\begin{bmatrix}\tauIT\\ \tauIIT\\ \tauIIIT\end{bmatrix},
\]
thus,
\[
\det(\B)=\begin{vmatrix}-1&0&0\\ 0&1&0\\ -\theta&0&1\end{vmatrix}
\begin{vmatrix}\tauIT\\ \tauIIT\\ \tauIIIT\end{vmatrix}=(-1)\cdot1=-1.
\]
Therefore, we know the system \eqref{eqn:error system} is a mixed-order FAS.

\section{Control strategy and stability analysis}

In this section, we design a 3-dimensional scalar estimator and a mixed-order FAS controller.
Then we prove the stability of this control strategy.

\subsection{Estimator and controller design}

In this subsection, to solve the 3-dimensional BOC problem, a control strategy consisting of 
a 3-dimension scalar estimator and a mixed-order FAS controller is designed.

Since the distance between the agent and the target is unknown, we need to design an estimator to locate the target.
Based on orthogonal projection method, there are two kinds of estimators in the BOC research area: the vector estimator
(\cite{ECDeghatShamesAndersonYu2010,ECDeghatDavisSee2012,EJDeghatShamesAndersonYu2014,
	EJDeghatXiaAndersonHong2015,
	EJLiShiSong2018,EJLiShiWuSong2019,EJYangZhuChenFengGuan2020,
	EJDouSongWangLiuFeng2020,ECMaShiLiWang2021}) 
and the scalar estimator
(\cite{EJCaoLiShiSong2020,EJCaoLiShiSong2021,ECCaoDuan2023,EJWangShiLi2024}).
In a nutshell, the difference of these two kinds of estimators is that the vector estimator estimates the target's position $\x$ but its scalar counterpart estimates the agent-target distance $\rho$.
In this paper, we adopt the scalar method and improve it to a 3-dimensional version.

Now we introduce our estimator.
Denote $\rhohat(t)\in\R$ as the estimated agent-target distance.
With this estimated distance, we can obtain the target's estimated position $\xhat(t)=\y(t)+\rhohat(t)\tauI(t)\in\R^3$.
Let $\tauIIiii(t)$ be the normalized vector of $\vII(t)\tauII(t)+\vIII(t)\tauIII(t)$,
that is,
\[
\tauIIiii(t)=\begin{cases}\displaystyle
	\frac{\vII(t)\tauII(t)+\vIII(t)\tauIII(t)}{\sqrt{\vII^2(t)+\vIII^2(t)}}&\text{if }\vII^2(t)+\vIII^2(t)\ne0\\
	0&\text{if }\vII^2(t)+\vIII^2(t)=0.
\end{cases}
\]
The 3-dimensional scalar estimator is designed as below:
\begin{equation}\label{eqn:estimator}
\rhohatdot=-\vI+k\tauIIiiiT\xhatdot.
\end{equation}
where $k\in\R_+$ is a coefficient.
The term ``$-\vI$'' is to compensate the inaccuracy caused by the agent’s movement.
The term ``$k\tauIIiiiT\xhatdot$'' is the derivation for negative feedback.

Let $\omega(t)=\sqrt{\omegaII^2+\omegaIII^2}\in\R$ be the agent's synthetic angular speed.
With this estimator, the dynamics of the agent-target distance estimation error 
$\rhotilde(t)=\rho(t)-\rhohat(t)\in\R$ is 
\begin{equation}\label{eqn:rhotilde dot=?}\begin{aligned}
\rhotildedot&=\rhodot-\rhohatdot=(-\vI)-(-\vI+k\tauIIiiiT\xhatdot)\\
&=-k\tauIIiiiT\xhatdot=-k\tauIIiiiT\ddt(\y+\rhohat\tauI)\\
&=-k\tauIIiiiT(\ydot+\rhohatdot\tauI+\rhohat\tauIdot)=-k\tauIIiiiT(\ydot+\rhohat\tauIdot)\\
&=-k\tauIIiiiT[(\vI\tauI+\vII\tauII+\vIII\tauIII)+\rhohat(-\omegaII\tauII-\omegaIII\tauIII)]\\
&=-k\tauIIiiiT[(\vII-\rhohat\omegaII)\tauII+(\vIII-\rhohat\omegaIII)\tauIII]\\
&=-k\tauIIiiiT[(\rho\omegaII-\rhohat\omegaII)\tauII+(\rho\omegaIII-\rhohat\omegaIII)\tauIII]\\
&=-k\tauIIiiiT[(\rhotilde\omegaII\tauII+\rhotilde\omegaIII\tauIII]
=-k\rhotilde\tauIIiiiT(\omegaII\tauII+\omegaIII\tauIII)\\
&=-k\rhotilde\tauIIiiiT(\omega\tauIIiii)=-k\omega\rhotilde\tauIIiiiT\tauIIiii=-k\omega\rhotilde.
\end{aligned}\end{equation}
The form of \eqref{eqn:rhotilde dot=?} is the same as that in the literature covering the 2-dimensional scalar estimator.
Hence, this newly designed 3-dimensional estimator should be considered as a natural extension of the 2-dimensional scalar estimator in the previous BOC papers.

Then we introduce the controller.
Let $\erhohat(t)=\rhohat(t)-d\in\R$ be the estimated agent-target distance tracking error.
Based on the mixed-order FAS theory (\cite{EJDuan2021_10-7}), 
the mixed-order FAS controller is designed as follows:
\begin{equation}\label{eqn:controller}
\yddot=-\B^{-1}\left(\bm f+\begin{bmatrix}
	-\arhoI\vI+\arhoO\erhohat\\ \avII\evII\\ \alI\ldian+\alO l
\end{bmatrix}\right).
\end{equation}
It can be noticed that, unlike usual treatments for FAS, the accurate error term ``$\erho$'' is replaced by the estimated one ``$\erhohat$''.
The reason lies in the particularity of the BOC problem, that is, 
the agent-target distance $\rho$ is unmeasurable.
This is exactly the difference between BOC FAS problems and regular FAS problems.

Finally, the constraints of the control coefficients are stated in the following assumption.

\begin{assumption}\label{asp:coefficients>0}
$\arhoI,\arhoO,\avII,\alI,\alO\in\R_+$.
\end{assumption}

\subsection{Stability analysis}

In the controller design, the exact error term ``$\erho$'' is replaced by 
its estimated counterpart ``$\erhohat$''.
In this section, we proved that this replacement does not influence the stability of the system.

Substituting the controller \eqref{eqn:controller} into the error system \eqref{eqn:error system},
we obtain the closed-loop error system equation as below,
\begin{subnumcases}{}
	\label{eqn:closed-loop error system equation - erho}
	\erhoddot+\arhoI\erhodot+\arhoO\erho=\arhoO\rhotilde\\
	\label{eqn:closed-loop error system equation - ev2}
	\evIIdot+\avII\evIIdot=0\\
	\label{eqn:closed-loop error system equation - l}
	\lddot+\alI\ldian+\alO l=0.
\end{subnumcases}
From \eqref{eqn:closed-loop error system equation - ev2}, \eqref{eqn:closed-loop error system equation - l} and Assumption~\ref{asp:coefficients>0}, we know the agent orbital tangential speed's tracking error $\evII$ and the arc length $l$ converges to zero exponentially.
In this subsection, we aim to prove the exponential convergence of the agent-target distance's tracking error $\erho$.

Firstly, the convergence of the estimator is proved.

\begin{lemma}\label{thm:rhotilde->0}
Under Assumption~\ref{asp:coefficients>0},
the estimator \eqref{eqn:estimator} and
the controller \eqref{eqn:controller}, it follows that
$\rhotilde$ exponentially converges to zero.
\end{lemma}

\begin{proof}
Considering \eqref{eqn:rhotilde dot=?}, let $V=\rhotilde^2/2$,
then $\dot V=\rhotilde\rhotildedot=-k\omega\rhotilde^2\le0$.
Thus we know $\rhotilde$ is bounded.
Considering \eqref{eqn:closed-loop error system equation - erho},
we know $\erho$ is bounded, hence $\rho=\erho+d$ is bounded.
Considering that $\vII$  converges to $\alpha$,
we know $\omegaII=\vII/\rho$ has a positive lower bound after some time.
Thus $\omega=\sqrt{\omega_2^2+\omega_3^2}$ has a positive lower bound after some time.
Denote $\mu\in\R_+$ as this lower bound.
So we know $\dot V<-k\mu V$ after some time.
Therefore, $\rhotilde$ exponentially converges to zero.
\end{proof}

To prove the convergence of the agent-target distance's tracking error $\erho$, the following proposition is shown.
\begin{proposition}[{\cite{ECCaoDuan2023}, Lemma~5}]\label{thm:from F1}
Consider the system
\[
\ddot w(t)+a_1\dot w(t)+a_0w(t)=z(t),
\]
	where $w(t)\in\R$; $a_1,a_0\in\R_+$; and $z(t)\in\R$ exponentially converges to zero.
	It follows that $w(t)$ exponentially converges to zero.
\end{proposition}

\begin{theorem}
Under Assumption~\ref{asp:coefficients>0},
the estimator \eqref{eqn:estimator} and
the controller \eqref{eqn:controller}, it follows that
$\erho$ exponentially converges to zero.
\end{theorem}

\begin{proof}
This is a natural conclusion from Lemma~\ref{thm:rhotilde->0}
and Proposition~\ref{thm:from F1}.
\end{proof}

Until now, we have proved that with our control strategy, the Problem~\ref{prb:BOC} has been solved.

\section{Simulation}

\subsection{Numerical simulation}

% TeX-master: "F3.tex"

% x, y and xhat, 3D trajectory
\begin{figure}[!htbp]
\centering
\begin{tikzpicture}\begin{axis}[
width=\linewidth,
axis equal image,
xmin=-12,xmax=22,ymin=-10,ymax=13,
xlabel={$x$-axis/m},every axis y label/.style={at={(ticklabel cs:0.5)},rotate=90,},
ylabel={$y$-axis/m},every axis x label/.style={at={(ticklabel cs:0.5,3pt)},},
xmajorgrids,ymajorgrids,
legend pos=south east,]
\addplot [color=red,mark=+] table[row sep=crcr] {0 0\\}; \addlegendentry{$\x$}
\addplot [color=green] table {fig/matlab/xhat.txt}; \addlegendentry{$\xhat$}
\addplot [color=blue] table {fig/matlab/y.txt}; \addlegendentry{$\y$}
\end{axis}\end{tikzpicture}
\caption{3D trajectories of $\y$ and $\xhat$.}\label{simfig:trajectory}
\end{figure}

% x, y and xhat, planar trajectory
\begin{figure}[!htbp]
\centering
\begin{tikzpicture}\begin{axis}[
width=\linewidth,
axis equal image,
xmin=-12,xmax=22,ymin=-10,ymax=13,
xlabel={$x$-axis/m},every axis y label/.style={at={(ticklabel cs:0.5)},rotate=90,},
ylabel={$y$-axis/m},every axis x label/.style={at={(ticklabel cs:0.5,3pt)},},
xmajorgrids,ymajorgrids,
legend pos=south east,]
\addplot [color=red,mark=+] table[row sep=crcr] {0 0\\}; \addlegendentry{$\x$}
\addplot [color=green] table {fig/matlab/xhat.txt}; \addlegendentry{$\xhat$}
\addplot [color=blue] table {fig/matlab/y.txt}; \addlegendentry{$\y$}
\end{axis}\end{tikzpicture}
\caption{Planar trajectories of $\y$ and $\xhat$.}\label{simfig:planar trajectory}
\end{figure}

% macro definitions
\def\signalfigurewidthscale{0.95}
\def\signalfigureheightscale{0.4}
\def\signalfigurexylabelset{
	every axis y label/.style={at={(-0.16,0.5)},rotate=90,},
	every axis x label/.style={at={(0.5,-0.5)},},}

% |xtilde|
\begin{figure}[!htbp]
\centering
\begin{tikzpicture}\begin{axis}[
width=\signalfigurewidthscale\linewidth,
height=\signalfigureheightscale\linewidth,
xmin=0,xmax=60,ymin=-10,ymax=5,\signalfigurexylabelset
xlabel={time/s},ylabel={$\rhotilde$\,/m},
xmajorgrids,ymajorgrids]
\addplot [color=blue] table {fig/matlab/eomega.txt};
\end{axis}\end{tikzpicture}
\caption{$\rhotilde$.}\label{simfig:rhotilde}
\end{figure}

% erho
\begin{figure}[!htbp]
\centering
\begin{tikzpicture}\begin{axis}[
width=\signalfigurewidthscale\linewidth,
height=\signalfigureheightscale\linewidth,
xmin=0,xmax=60,ymin=-2,ymax=18,\signalfigurexylabelset
xlabel={time/s},ylabel={$\erho$\,/m},
xmajorgrids,ymajorgrids]
\addplot [color=blue] table {fig/matlab/eomega.txt};
\end{axis}\end{tikzpicture}
\caption{$\erho$.}\label{simfig:erho}
\end{figure}

% ev2
\begin{figure}[!htbp]
\centering
\begin{tikzpicture}\begin{axis}[
width=\signalfigurewidthscale\linewidth,
height=\signalfigureheightscale\linewidth,
xmin=0,xmax=60,ymin=-0.3,ymax=0.1,\signalfigurexylabelset
xlabel={time/s},ylabel={$\evII$\,/m},
xmajorgrids,ymajorgrids]
\addplot [color=blue] table {fig/matlab/eomega.txt};
\end{axis}\end{tikzpicture}
\caption{$\evII$.}\label{simfig:ev2}
\end{figure}

% l
\begin{figure}[!htbp]
\centering
\begin{tikzpicture}\begin{axis}[
width=\signalfigurewidthscale\linewidth,
height=\signalfigureheightscale\linewidth,
xmin=0,xmax=60,ymin=-0.3,ymax=0.1,\signalfigurexylabelset
xlabel={time/s},ylabel={$l$\,/m},
xmajorgrids,ymajorgrids]
\addplot [color=blue] table {fig/matlab/eomega.txt};
\end{axis}\end{tikzpicture}
\caption{$l$.}\label{simfig:l}
\end{figure}


\subsection{PX4 software-in-the-loop simulation}

%% TeX-master: "F3.tex"

% x, y and xhat, 3D trajectory
\begin{figure}[!htbp]
\centering
\begin{tikzpicture}\begin{axis}[
width=\linewidth,
axis equal image,
xmin=-12,xmax=22,ymin=-10,ymax=13,
xlabel={$x$-axis/m},every axis y label/.style={at={(ticklabel cs:0.5)},rotate=90,},
ylabel={$y$-axis/m},every axis x label/.style={at={(ticklabel cs:0.5,3pt)},},
xmajorgrids,ymajorgrids,
legend pos=south east,]
\addplot [color=red,mark=+] table[row sep=crcr] {0 0\\}; \addlegendentry{$\x$}
\addplot [color=green] table {fig/matlab/xhat.txt}; \addlegendentry{$\xhat$}
\addplot [color=blue] table {fig/matlab/y.txt}; \addlegendentry{$\y$}
\end{axis}\end{tikzpicture}
\caption{3D trajectories of $\y$ and $\xhat$.}\label{simfig:trajectory}
\end{figure}

% x, y and xhat, planar trajectory
\begin{figure}[!htbp]
\centering
\begin{tikzpicture}\begin{axis}[
width=\linewidth,
axis equal image,
xmin=-12,xmax=22,ymin=-10,ymax=13,
xlabel={$x$-axis/m},every axis y label/.style={at={(ticklabel cs:0.5)},rotate=90,},
ylabel={$y$-axis/m},every axis x label/.style={at={(ticklabel cs:0.5,3pt)},},
xmajorgrids,ymajorgrids,
legend pos=south east,]
\addplot [color=red,mark=+] table[row sep=crcr] {0 0\\}; \addlegendentry{$\x$}
\addplot [color=green] table {fig/matlab/xhat.txt}; \addlegendentry{$\xhat$}
\addplot [color=blue] table {fig/matlab/y.txt}; \addlegendentry{$\y$}
\end{axis}\end{tikzpicture}
\caption{Planar trajectories of $\y$ and $\xhat$.}\label{simfig:planar trajectory}
\end{figure}

% macro definitions
\def\signalfigurewidthscale{0.95}
\def\signalfigureheightscale{0.4}
\def\signalfigurexylabelset{
	every axis y label/.style={at={(-0.16,0.5)},rotate=90,},
	every axis x label/.style={at={(0.5,-0.5)},},}

% |xtilde|
\begin{figure}[!htbp]
\centering
\begin{tikzpicture}\begin{axis}[
width=\signalfigurewidthscale\linewidth,
height=\signalfigureheightscale\linewidth,
xmin=0,xmax=60,ymin=-10,ymax=5,\signalfigurexylabelset
xlabel={time/s},ylabel={$\rhotilde$\,/m},
xmajorgrids,ymajorgrids]
\addplot [color=blue] table {fig/matlab/eomega.txt};
\end{axis}\end{tikzpicture}
\caption{$\rhotilde$.}\label{simfig:rhotilde}
\end{figure}

% erho
\begin{figure}[!htbp]
\centering
\begin{tikzpicture}\begin{axis}[
width=\signalfigurewidthscale\linewidth,
height=\signalfigureheightscale\linewidth,
xmin=0,xmax=60,ymin=-2,ymax=18,\signalfigurexylabelset
xlabel={time/s},ylabel={$\erho$\,/m},
xmajorgrids,ymajorgrids]
\addplot [color=blue] table {fig/matlab/eomega.txt};
\end{axis}\end{tikzpicture}
\caption{$\erho$.}\label{simfig:erho}
\end{figure}

% ev2
\begin{figure}[!htbp]
\centering
\begin{tikzpicture}\begin{axis}[
width=\signalfigurewidthscale\linewidth,
height=\signalfigureheightscale\linewidth,
xmin=0,xmax=60,ymin=-0.3,ymax=0.1,\signalfigurexylabelset
xlabel={time/s},ylabel={$\evII$\,/m},
xmajorgrids,ymajorgrids]
\addplot [color=blue] table {fig/matlab/eomega.txt};
\end{axis}\end{tikzpicture}
\caption{$\evII$.}\label{simfig:ev2}
\end{figure}

% l
\begin{figure}[!htbp]
\centering
\begin{tikzpicture}\begin{axis}[
width=\signalfigurewidthscale\linewidth,
height=\signalfigureheightscale\linewidth,
xmin=0,xmax=60,ymin=-0.3,ymax=0.1,\signalfigurexylabelset
xlabel={time/s},ylabel={$l$\,/m},
xmajorgrids,ymajorgrids]
\addplot [color=blue] table {fig/matlab/eomega.txt};
\end{axis}\end{tikzpicture}
\caption{$l$.}\label{simfig:l}
\end{figure}





\section{Conclusion and future work}











\section{Acknowledgements}
This work is supported by
Science Center Program of the National Natural
Science Foundation of China (Grant No. 62188101).



\bibliographystyle{elsarticle-harv} 
\bibliography{caobibfile}


\end{document}

\endinput


